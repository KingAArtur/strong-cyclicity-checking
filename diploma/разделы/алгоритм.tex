\section{Разработка модифицированного алгоритма}

\usepackage{listings}
\begin{lstlisting}[language=Python]

На базе описанного метода был реализован модифицированный алгоритм, использующий рекурсию. В отличие от предыдущего алгоритма, который выбирает для продолжения лишь один элемент матрицы, модифицированный алгоритм будет перебирать все возможные элементы для продолжения.
Вход алгоритма:
квадратная матрица A размера mm с полиномиальными коэффициентами
циклический вектор v
максимальная глубина рекурсии depth
Выход алгоритма:
ответ “не сильно циклический” + продолжение, нарушающее сильную цикличность;
ответ “сильно циклический”;
ответ “неизвестно”.
Шаги алгоритма:
Если depth = 0, вернуть ответ “неизвестно”.
Положить solutions = пустой список.
for i from 1 to m:
for j from 1 to m:
Скопировать матрицу A в матрицу B;
Продолжить элемент матрицы Bij с символьным коэффициентом c;
Построить для B матрицу [Bv], вычислить ее определитель det(x, c) как полином от переменной x;
Найти все ненулевые значения для c, зануляющие трейлинговый коэффициент det(x, c) и запомнить их в переменную roots
for s in roots:
если det(x, s) = 0, то вернуть ответ “не сильно циклический” и данное продолжение c числовым коэффициентом s,
иначе добавить значение s и индексы i,j в список solutions;
for s, (i, j) in solutions:
Скопировать матрицу A в матрицу B;
Продолжить элемент матрицы Bij с числовым коэффициентом s;
Рекурсивно применить алгоритм к матрице B c тем же вектором v и глубиной рекурсии depth - 1;
Если получен ответ “не сильно циклический” + продолжение, вернуть данный ответ и продолжение.
Если на предыдущем шаге хотя бы для одного продолжения был получен ответ “неизвестно”, вернуть ответ “неизвестно”.
Вернуть ответ “сильно циклический”.
Как и в случае исходного алгоритма, возможна ситуация, при которой степени слагаемых, входящих в определитель матрицы производных в силу системы, будут неограниченно расти. В этом случае при достижении максимальной глубины рекурсии алгоритм будет вынужден дать ответ “неизвестно”. Однако, шанс возникновения такой ситуации существенно ниже, потому что для продолжения перебираются все возможные элементы матрицы.