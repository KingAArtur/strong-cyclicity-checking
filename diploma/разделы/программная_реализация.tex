\section{Программная реализация метода}

В этом разделе мы рассмотрим программную реализацию данного проекта.
6.1  Система компьютерной алгебры Мэйпл
Мэйпл является программной системой для работы с компьютерной алгеброй [5]. Мэйпл предоставляет колоссальное количество возможностей для математических вычислений, визуализации данных и моделирования.
Когда пользователь запускает систему и создает новый документ, на экране появляется рабочая область, в который он может вводить формулы, выражения и т.д. Это выглядит следующим образом:

Рис. 6.1. Рабочая область в Мэйпл
Рабочая область состоит из последовательных ячеек, содержащих обычный текст или математические выражения. По команде пользователя все содержимое ячейки может быть выполнено, при этом результаты вычисления появятся под ячейкой. Эти результаты могут быть использованы в дальнейших расчетах.
Мэйпл предоставляет свой собственный язык программирования. Язык может быть использован для реализации своих собственных функций, процедур и структур данных. Присутствуют все обычные элементы императивной парадигмы программирования: циклы, присваивания, переменные, условные выражения, списки и пр. Реализованные процедуры могут быть сохранены в текстовый файл для дальнейшего использования.

6.2  Основные функции и структуры данных
\text{
В рамках данной работы было реализовано несколько функций. Из них выделяются следующие основные функции (исходный код приведен в Приложении А):
Функция для построения продолжения многочлена
prolong := proc(q, c, need_indexes, i, j)
q – многочлен от переменной x, возможно, содержащий слагаемое O(xk) для некоторого целого k
c – символьный или числовой коэффициент
need_indexes (необязательный параметр) – булево значение, указывающее, нужно ли приписывать к символьному коэффициенту индекс, характеризующий степень добавленного слагаемого
i, j (необязательные параметры) – индексы, которые необходимо приписать к символьному коэффициенту
Данная функция возвращает новый многочлен, являющийся продолжением многочлена q(x). Продолжение получается добавлением нового слагаемого ct,i,jxt, где t вычисляется следующим образом:
если q(x)=0, то t=0
если q(x) – многочлен степени m без слагаемого O(xk), то t=m+1
если q(x) – многочлен со слагаемым O(xk), то t=k
Примеры использования:

Рис. 6.2.1. Пример использования функции prolong
Функция для построения матрицы производных в силу системы
CV_dv := proc(A, v, m)
A – квадратная матрица размера nn с коэффициентами, являющимися полиномами от независимой переменной x
v – вектор длины n с коэффициентами, являющимися полиномами от независимой переменной x
m – количество столбцов результирующей матрицы, которое нужно вычислить (по умолчанию равно n) 
Данная функция возвращает матрицу, столбцами которой являются v, Av, A2v, ..., Am-1v. При m=n эта матрица будет являться матрицей производных в силу системы для матрицы A и вектора v, которая может использоваться для определения того, является ли вектор v циклическим.
Главная процедура, реализующая алгоритм для проверки сильной цикличности вектора
is_strong_cyclic := proc(A, v, step)
A – квадратная матрица размера nn с коэффициентами, являющимися полиномами от независимой переменной x
v – вектор длины n с коэффициентами, являющимися полиномами от независимой переменной x
step – максимальная глубина рекурсии (по умолчанию 10)
Пример работы:

Рис 6.2.3. Пример использования функции is_strong_cyclic
6.3  Скачивание и использование
Реализация алгоритма доступна в виде набора процедур для Мэйпл по следующей ссылке: https://gitlab.com/AAArtur/strong-cyclicity-checking.}