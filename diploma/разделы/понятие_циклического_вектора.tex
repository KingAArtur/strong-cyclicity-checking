\section{Понятие циклического вектора}

Пусть $K$ "--- поле характеристики нуль (например, поле рациональных чисел),
$K[x]$ "--- кольцо \emph{многочленов} переменной~$x$,
$K[[x]]$ "--- кольцо \emph{формальных степенных рядов} от~независимой переменной~$x$:
\[
	K[x] = \{a(x) = \sum\limits_{n = 0}^t a_n x^n, a_n \in K, t \in \mathbb{N}_0\}
\]

\[
	K[[x]] = \{a(x) = \sum\limits_{n = 0}^\infty a_n x^n, a_n \in K\}.
\]
\medskip

Будем рассматривать дифференциальные системы вида
\begin{equation}
	y' = Ay,
	\label{system}
\end{equation}
где $A$ "--- квадратная матрица размера~$m \times m$ коэффициентов системы, имеющих вид формальных степенных рядов,
$y$ "--- вектор неизвестных из~$m$ компонент.

Пусть $v \in K[x]^m$ "--- вектор из~$m$ компонент "--- многочленов. По аналогии с~\cite{litVanDerPut}, введем оператор вида
\begin{equation}
    \diff: v\rightarrow v' + A^T v,
    \label{differention}
\end{equation}
где $A$ "--- матрица коэффициентов системы \eqref{system}, $v'$ "--- покомпонентное дифференцирование вектора~$v$.
Будем говорить, что этот оператор задает \emph{дифференцирование вектора~$v$ в~силу системы}~\eqref{system}.

\begin{example}
	Пусть имеются матрица $A$ и вектор $v$:
	\begin{equation*}
		A = 
		\begin{pmatrix}
			-2 & x^2 & 1 \\
			3 & x & 2x^2 \\
			4 & x^3 & x \\
		\end{pmatrix},\quad
		v =
		\begin{pmatrix}
			1 \\
			x \\
			3x + 2x^2 \\
		\end{pmatrix}.
	\end{equation*}
    
	Продифференцируем вектор $v$ в силу системы с матрицей $A$:
	\begin{equation*}
		\diffVector = v' + A^T v = 
		\begin{pmatrix}
			0 \\
			1 \\
			3 + 4x \\
		\end{pmatrix}
        +
        \begin{pmatrix}
			-2 & 3 & 4 \\
			x^2 & x & x^3 \\
			1 & 2x^2 & x \\
		\end{pmatrix}
        \times
        \begin{pmatrix}
			1 \\
			x \\
			3x + 2x^2 \\
		\end{pmatrix}
        =
    \end{equation*}
    
    \begin{equation*}
        =
        \begin{pmatrix}
			0 \\
			1 \\
			3 + 4x \\
		\end{pmatrix}
        +
		\begin{pmatrix}
			8x^2 + 15x - 2 \\
			2x^5 + 3x^4 + 2x^2 \\
			4x^3 + 3x^2  + 1 \\
		\end{pmatrix} =
		\begin{pmatrix}
			8x^2 + 15x - 2 \\
			2x^5 + 3x^4 + 2x^2 + 1\\
			4x^3 + 3x^2 + 4x + 4 \\
		\end{pmatrix}.
	\end{equation*}
\end{example}
\bigskip

\newpage
\emph{Матрицей производных вектора~$v$ в силу системы} \eqref{system} назовем квадратную матрицу размера~$m \times m$,
в которой по столбцам записаны векторы $v$, \diffVector, \diffVector[2], \dots, \diffVector[m-1]:

\begin{equation}
	\diffMatrix = [v \mid \diffVector \mid \diffVector[2] \mid \dots \mid \diffVector[m-1]]
    \label{diffMatrix}
\end{equation}

\begin{example}
	Пусть имеются матрица $A$ и вектор $v$:
	\begin{equation*}
		A = 
		\begin{pmatrix}
			1 & x & 3 \\
			2x & 1 & 0 \\
			3 & -1 & x^2 \\
		\end{pmatrix},\quad
		v =
		\begin{pmatrix}
			1 \\
			0 \\
			0 \\
		\end{pmatrix}.
	\end{equation*}
    
	Матрица производных вектора $v$ в силу системы выглядит следующим образом:
	\begin{equation*}
		\diffMatrix = [v \mid \diffVector \mid \diffVector[2]] = 
		\begin{pmatrix}
			1 & 1 & 2x^2 + 10 \\
			0 & x & -2 + 2x \\
			0 & 3 & 3x^2 + 3 \\
		\end{pmatrix}
	\end{equation*}
\end{example}

Вектор~$v$ называется \emph{циклическим для~системы} \eqref{system}, если векторы $v$, \diffVector, \diffVector[2], \dots, \diffVector[m-1] линейно независимы.
С~помощью циклического вектора задачу решения системы дифференциальных уравнений можно свести к~задаче решения скалярного дифференциального уравнения.

Определение циклического вектора можно записать и в~матричном виде:
вектор~$v$ является циклическим, если и только если
\begin{equation*}
	\dett(\diffMatrix) \neq 0.
%	\label{diffMatrixNotZero}
\end{equation*}