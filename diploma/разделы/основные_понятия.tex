\section{Основные понятия}

Пусть $K$~---~поле характеристики нуль,  $K[x]$~---~кольцо \emph{многочленов} переменной~$x$,  $K[[x]]$~---~кольцо \emph{формальных степенных рядов} от независимой переменной~$x$:
\[
	K[x] = \{a(x) = \sum\limits_{n = 0}^t a_n x^n, a_n \in K, t \in \mathbb{N}_0\}
\]

\[
	K[[x]] = \{a(x) = \sum\limits_{n = 0}^\infty a_n x^n, a_n \in K\}.
\]

Коэффициент при~наименьшей степени~$x$ в многочлене называется \emph{трейлинговым коэффициентом}, а степень при~этом коэффициенте~---~\emph{валюацией}.
Например, для многочлена $3x^2 + 4x^5 +x^8$ валюация $\val(3x^2 + 4x^5 +x^8)$ равна~$2$, а трейлинговый коэффициент $\tc(3x^2 + 4x^5 +x^8)$ равен~$3$.

Для заданного числа
$l \in \mathbb{Z}$
и ряда
$a(x) \in K[[x]]$
определим \emph{$l$-усечение ряда} как многочлен, который получается из~$a(x)$ занулением всех коэффициентов при~степенях~$x$, больших~$l$:
\[
	a^{\langle l \rangle}(x) = \sum\limits_{n = 0}^l a_nx^n .
\]
Таким образом, $l$-усечение~$a^{\langle l \rangle}(x)$ является многочленом, степень которого не~превосходит~$l$.


\emph{Продолжением} заданного многочлена~$a(x)$ степени~$l$ будем называть любой многочлен~$b(x)$,
$l$-усечение которого совпадает с~$a(x)$: $b^{\langle l \rangle}(x) = a(x)$.

Например, для ряда $ a(x) = 3x + 2x^2 + 5x^4 + x^8 + \dots $ $2$-усечением будет являться многочлен $ a^{\langle 2 \rangle}(x) = 3x + 2x^2 $.
А многочлен $ b(x) = 3x + 2x^2 + 4x^3 + 2x^{11} $ является возможным продолжением многочлена $a^{\langle 2 \rangle}(x)$.

\bigskip
Будем рассматривать дифференциальные системы вида
\begin{equation}
	y' = Ay,
	\label{system}
\end{equation}
где $A$ --- квадратная матрица размера~$m \times m$ коэффициентов системы, имеющих вид формальных степенных рядов,
$y$~---~вектор неизвестных размера~$m$, компонентами которого являются функции~$y(x)$.

Пусть $v \in K[x]^m$ --- вектор из~$m$ компонент~---~многочленов. Введем~\cite{litVanDerPut} оператор вида
\begin{equation}
\diff: v\rightarrow v' + A^T v,
\label{differention}
\end{equation}
где $A$ --- матрица коэффициентов системы \rref{system}, $v'$~---~покомпонентное дифференцирование вектора~$v$.
Будем говорить, что этот оператор задает \emph{дифференцирование вектора $v$ в силу системы}~\rref{system}.

\begin{example}
	Пусть
	\begin{equation*}
		A = 
		\begin{pmatrix}
			-2 & x^2 & 1 \\
			3 & x & 2x^2 \\
			4 & x^3 & x \\
		\end{pmatrix},
		v =
		\begin{pmatrix}
			1 \\
			x \\
			3x + 2x^2 \\
		\end{pmatrix}.
	\end{equation*}
    
	Тогда
	\begin{equation*}
		\diffVector = v' + A^T v = 
		\begin{pmatrix}
			0 \\
			1 \\
			3 + 4x \\
		\end{pmatrix}
        +
        \begin{pmatrix}
			-2 & 3 & 4 \\
			x^2 & x & x^3 \\
			1 & 2x^2 & x \\
		\end{pmatrix}
        \times
        \begin{pmatrix}
			1 \\
			x \\
			3x + 2x^2 \\
		\end{pmatrix}
        =
    \end{equation*}
    
    \begin{equation*}
        =
        \begin{pmatrix}
			0 \\
			1 \\
			3 + 4x \\
		\end{pmatrix}
        +
		\begin{pmatrix}
			8x^2 + 15x - 2 \\
			2x^5 + 3x^4 + 2x^2 \\
			4x^3 + 3x^2  + 1 \\
		\end{pmatrix} =
		\begin{pmatrix}
			8x^2 + 15x - 2 \\
			2x^5 + 3x^4 + 2x^2 + 1\\
			4x^3 + 3x^2 + 4x + 4 \\
		\end{pmatrix}
	\end{equation*}
\end{example}
\bigskip

Вектор $v$ называется \emph{циклическим для системы} \rref{system}, если векторы $v$, \diffVector, \diffVector[2], \dots, \diffVector[m-1] линейно независимы.
С помощью циклического вектора задачу решения системы дифференциальных уравнений можно свести к задаче решения скалярного дифференциального уравнения.

\emph{Матрицей производных вектора $v$ в силу системы} \rref{system} назовем квадратную матрицу размера~$m \times m$,
в которой по столбцам записаны векторы $v$, \diffVector, \diffVector[2], \dots, \diffVector[m-1]:

\begin{equation}
	\diffMatrix = [v \mid \diffVector \mid \diffVector[2] \mid \dots \mid \diffVector[m-1]]
	\label{diffMatrix}
\end{equation}

\begin{example}
	Пусть
	\begin{equation*}
		A = 
		\begin{pmatrix}
			1 & x & 3 \\
			2x & 1 & 0 \\
			3 & -1 & x^2 \\
		\end{pmatrix},
		v =
		\begin{pmatrix}
			1 \\
			0 \\
			0 \\
		\end{pmatrix}.
	\end{equation*}
    
	Тогда
	\begin{equation*}
		\diffMatrix = [v \mid \diffVector \mid \diffVector[2]] = 
		\begin{pmatrix}
			1 & 1 & 2x^2 + 10 \\
			0 & x & -2 + 2x \\
			0 & 3 & 3x^2 + 3 \\
		\end{pmatrix}
	\end{equation*}
\end{example}

Таким образом, определение циклического вектора можно переписать в матричном виде:
вектор~$v$ является циклическим, если и только если
\begin{equation}
	\dett(\diffMatrix) \neq 0
	\label{diffMatrixNotZero}
\end{equation}
%\bigskip

Ряды по своей природе являются \emph{бесконечными} структурами, что вызывает сложности при работе с ними на компьютере.
Одним из способов решения данной проблемы является переход к работе с усечениями.
Так как компьютерная алгебра находит все большее распространение,
методы работы с усечениями становятся все более и более востребованными.
\medskip

Таким образом, мы приходим к тому, что исходная дифференциальная система нам известна не полностью.
Вместо этого нам дана \emph{усеченная система}~---~в ней коэффициенты исходной системы (являющиеся формальными степенными рядами) представлены
усечениями рядов~---~многочленами, и часть информации об исходной системе теряется.

Для простоты изложения будем считать, что все коэффициенты имеют одну и ту же степень усечения,
которую мы будем называть \emph{степенью усечения системы}.
Продолжая некоторым образом коэффициенты усеченной системы, можно получить \emph{продолжение} усеченной системы.
Продолжение может как совпадать с исходной дифференциальной системой, так и отличаться от нее.
\medskip

Данная работа посвящена циклическим векторам.
Для конкретной усеченной системы понять, является ли данный вектор циклическим, довольно легко:
достаточно проверить невырожденность матрицы производных данного вектора в силу системы.
Однако вектор, циклический для данной усеченной системы, может не оказаться циклическим для исходной системы.

Поскольку необходимая информация об исходной системе отсутствует,
логичным решением будет исследовать цикличность вектора для всех возможных продолжений системы.
Циклический вектор, который является циклическим для любого возможного продолжения данной системы, называется \emph{сильно циклическим}.

В случае, когда удается доказать, что данный вектор является сильно циклическим для заданной усеченной системы,
мы можем утверждать, что он является циклическим не только для усеченной системы, но и для исходной системы.
Это означает, что мы можем применить некоторые методы к заданной усеченной системе и получить результат,
согласующийся с исходной дифференциальной системой.
Проверка того, является ли заданный вектор сильно циклическим, представляет основной интерес данной работы.
