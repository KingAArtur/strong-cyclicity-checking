\section{Основные понятия}

Пусть $K$ "--- поле характеристики нуль,
$K[x]$ "--- кольцо \emph{многочленов} переменной~$x$,
$K[[x]]$ "--- кольцо \emph{формальных степенных рядов} от независимой переменной~$x$:
\[
	K[x] = \{a(x) = \sum\limits_{n = 0}^t a_n x^n, a_n \in K, t \in \mathbb{N}_0\}
\]

\[
	K[[x]] = \{a(x) = \sum\limits_{n = 0}^\infty a_n x^n, a_n \in K\}.
\]

Коэффициент при~наименьшей степени~$x$ в многочлене называется \emph{трейлинговым коэффициентом},
а степень при~этом коэффициенте "--- \emph{валюацией}.
Например, для~многочлена $3x^2 + 4x^5 +x^8$ валюация $\val(3x^2 + 4x^5 +x^8)$ равна~$2$,
а трейлинговый коэффициент $\tc(3x^2 + 4x^5 +x^8)$ равен~$3$.

Для~заданного числа
$\ell \in \mathbb{N}_0$
и ряда
$a(x) \in K[[x]]$
определим \emph{$\ell$-усечение ряда} как многочлен, который получается из~$a(x)$
занулением всех коэффициентов при~степенях~$x$, больших~$\ell$:
\[
	a^{\langle \ell \rangle}(x) = \sum\limits_{n = 0}^{\ell} a_nx^n .
\]
Таким образом, $\ell$-усечение~$a^{\langle \ell \rangle}(x)$ является многочленом, степень которого не~превосходит~$\ell$.


\emph{Продолжением} заданного многочлена~$a(x)$ степени~$\ell$ будем называть любой многочлен~$b(x)$,
$\ell$-усечение которого совпадает с~$a(x)$: $b^{\langle \ell \rangle}(x) = a(x)$.

Например, для ряда $ a(x) = 3x + 2x^2 + 5x^4 + x^8 + \dots $ $2$-усечением будет являться многочлен $ a^{\langle 2 \rangle}(x) = 3x + 2x^2 $.
А многочлен $ b(x) = 3x + 2x^2 + 4x^3 + 2x^{11} $ является одним из возможных продолжений многочлена $a^{\langle 2 \rangle}(x)$.

\bigskip
Будем рассматривать дифференциальные системы вида
\begin{equation}
	y' = Ay,
	\label{system}
\end{equation}
где $A$ "--- квадратная матрица размера~$m \times m$ коэффициентов системы, имеющих вид формальных степенных рядов,
$y$ "--- вектор неизвестных размера~$m$, компонентами которого являются функции~$y(x)$.

Пусть $v \in K[x]^m$ "--- вектор из~$m$ компонент "--- многочленов. Введем~\cite{litVanDerPut} оператор вида
\begin{equation}
    \diff: v\rightarrow v' + A^T v,
    \label{differention}
\end{equation}
где $A$ "--- матрица коэффициентов системы \eqref{system}, $v'$ "--- покомпонентное дифференцирование вектора~$v$.
Будем говорить, что этот оператор задает \emph{дифференцирование вектора~$v$ в~силу системы}~\eqref{system}.

\begin{example}
	Пусть
	\begin{equation*}
		A = 
		\begin{pmatrix}
			-2 & x^2 & 1 \\
			3 & x & 2x^2 \\
			4 & x^3 & x \\
		\end{pmatrix},
		v =
		\begin{pmatrix}
			1 \\
			x \\
			3x + 2x^2 \\
		\end{pmatrix}.
	\end{equation*}
    
	Тогда
	\begin{equation*}
		\diffVector = v' + A^T v = 
		\begin{pmatrix}
			0 \\
			1 \\
			3 + 4x \\
		\end{pmatrix}
        +
        \begin{pmatrix}
			-2 & 3 & 4 \\
			x^2 & x & x^3 \\
			1 & 2x^2 & x \\
		\end{pmatrix}
        \times
        \begin{pmatrix}
			1 \\
			x \\
			3x + 2x^2 \\
		\end{pmatrix}
        =
    \end{equation*}
    
    \begin{equation*}
        =
        \begin{pmatrix}
			0 \\
			1 \\
			3 + 4x \\
		\end{pmatrix}
        +
		\begin{pmatrix}
			8x^2 + 15x - 2 \\
			2x^5 + 3x^4 + 2x^2 \\
			4x^3 + 3x^2  + 1 \\
		\end{pmatrix} =
		\begin{pmatrix}
			8x^2 + 15x - 2 \\
			2x^5 + 3x^4 + 2x^2 + 1\\
			4x^3 + 3x^2 + 4x + 4 \\
		\end{pmatrix}
	\end{equation*}
\end{example}
\bigskip

\emph{Матрицей производных вектора~$v$ в силу системы} \eqref{system} назовем квадратную матрицу размера~$m \times m$,
в которой по столбцам записаны векторы $v$, \diffVector, \diffVector[2], \dots, \diffVector[m-1]:

\begin{equation}
	\diffMatrix = [v \mid \diffVector \mid \diffVector[2] \mid \dots \mid \diffVector[m-1]]
    \label{diffMatrix}
\end{equation}

\begin{example}
	Пусть
	\begin{equation*}
		A = 
		\begin{pmatrix}
			1 & x & 3 \\
			2x & 1 & 0 \\
			3 & -1 & x^2 \\
		\end{pmatrix},
		v =
		\begin{pmatrix}
			1 \\
			0 \\
			0 \\
		\end{pmatrix}.
	\end{equation*}
    
	Тогда
	\begin{equation*}
		\diffMatrix = [v \mid \diffVector \mid \diffVector[2]] = 
		\begin{pmatrix}
			1 & 1 & 2x^2 + 10 \\
			0 & x & -2 + 2x \\
			0 & 3 & 3x^2 + 3 \\
		\end{pmatrix}
	\end{equation*}
\end{example}

Вектор~$v$ называется \emph{циклическим для~системы} \eqref{system}, если векторы $v$, \diffVector, \diffVector[2], \dots, \diffVector[m-1] линейно независимы.
С~помощью циклического вектора задачу решения системы дифференциальных уравнений можно свести к~задаче решения скалярного дифференциального уравнения.

Определение циклического вектора можно записать и в~матричном виде:
вектор~$v$ является циклическим, если и только если
\begin{equation}
	\dett(\diffMatrix) \neq 0
	\label{diffMatrixNotZero}
\end{equation}

Ряды по~своей природе являются \emph{бесконечными} структурами, что вызывает сложности при~работе с~ними на~компьютере.
Одним из~способов решения данной проблемы является переход к~работе с~усечениями.
Так~как компьютерная алгебра находит все большее распространение,
методы работы с~усечениями становятся все~более и более востребованными.
\medskip

Применительно к~дифференциальным уравнениям данный подход впервые был~предложен С.\,А.~Абрамовым и его соавторами.
В~своих работах \cite{litAbramovTruncatedSeries, litAbramovScalarEquations}
они рассматривали скалярные дифференциальные уравнения,
коэффициенты которых были представлены усечениями степенных рядов.
Они интересовались получением такой информации о~решениях,
которая была~бы инвариантна относительно возможных продолжений коэффициентов уравнения.

Таким образом, мы приходим к~тому, что исходная дифференциальная система нам известна не~полностью.
Вместо этого нам дана \emph{усеченная система} "--- в~ней коэффициенты исходной системы (являющиеся формальными степенными рядами) представлены
усечениями рядов "--- многочленами, и часть информации об~исходной системе теряется.

Для простоты изложения будем считать, что все коэффициенты имеют одну и ту~же степень усечения,
которую мы будем называть \emph{степенью усечения системы}.
Продолжая некоторым образом коэффициенты усеченной системы, можно получить \emph{продолжение} усеченной системы.
Продолжение может как~совпадать с~исходной дифференциальной системой, так и отличаться от~нее.

Основной операцией в предлагаемом далее алгоритме будет построение \emph{одноэлементного продолжения} матрицы.
Оно заключается в~прибавлении к~некоторому элементу матрицы слагаемого вида $c \cdot x^t$,
где~$c$ "--- некоторый символьный или числовой коэффициент,
$t$ "--- степень, которая, как правило, на~единицу больше степени элемента матрицы как~многочлена.

Введем обозначение
\begin{equation*}
	I_{ij} = 
	\bordermatrix{
		&           &         &        & j      &        &        \cr
		&   0       & 0       & \cdots & 0      & \cdots & 0      \cr
		&   \vdots  & \vdots  & \ddots & \vdots & \ddots & \vdots \cr
		i & 0       & 0       & \cdots & 1      & \cdots & 0      \cr
		&   \vdots  & \vdots  & \ddots & \vdots & \ddots & \vdots \cr
		&   0       & 0       & \cdots & 0      & \cdots & 0      \cr
	}.
\end{equation*}
$I_{ij}$ "--- матрица, у~которой элемент в~позиции~$[i, j]$ равен~$1$, а все остальные элементы равны нулю.
Одноэлементное продолжение матрицы теперь можно кратко записать следующим образом:
\begin{equation*}
	B = A + cx^t \cdot I_{ij}.
\end{equation*}

Данная работа посвящена циклическим векторам.
Для~конкретной усеченной системы понять, является~ли данный вектор циклическим, довольно легко:
достаточно проверить невырожденность матрицы производных данного вектора в~силу системы.
Однако вектор, циклический для~данной усеченной системы, может не~оказаться циклическим для~исходной системы.

\newpage
Поскольку необходимая информация об~исходной системе отсутствует,
логичным решением будет исследовать цикличность вектора для~всех возможных продолжений системы.
Циклический вектор, который является циклическим для~любого возможного продолжения данной системы,
называется \emph{сильно циклическим}.

\begin{example}
Пусть
	\begin{equation*}
		A = 
		\begin{pmatrix}
			1 & 1 & x + 1 \\
			1 & x - 1 & -x + 1 \\
			-x & -5x + 2 & x - 1 \\
		\end{pmatrix},
		v =
		\begin{pmatrix}
			1 \\
			0 \\
			0 \\
		\end{pmatrix}.
	\end{equation*}
    
	Тогда вектор~$v$ \textbf{не}~является сильно циклическим для~системы с~матрицей~$A$,
    поскольку существует продолжение~$B$ матрицы~$A$ такое,
    что для~него вектор уже не~является циклическим:
	\begin{equation*}
		B = 
		\begin{pmatrix}
			1 & 1 & x + 1 \\
			1 & x - 1 & -5x^3 -8x^2 -x + 1 \\
			-x & -5x + 2 & x - 1 \\
		\end{pmatrix},
	\end{equation*}
    \begin{equation*}
		\dett\diffMatrix[B] = 0.
	\end{equation*}
\end{example}

В~случае, когда удается доказать, что данный вектор является сильно циклическим для~заданной усеченной системы,
мы можем утверждать, что он является циклическим не~только для~усеченной системы, но и для~исходной системы.
Это означает, что мы можем использовать данный циклический вектор и получить результат,
согласующийся с~исходной дифференциальной системой.
Проверка того, является~ли заданный вектор сильно циклическим, представляет основной интерес данной работы.
