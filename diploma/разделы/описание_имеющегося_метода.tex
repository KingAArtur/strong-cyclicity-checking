\section{Описание имеющегося метода}

Имеющийся алгоритм описан в [4]. Он основан на операции продолжения элемента матрицы, т. е. на переходе от исходной матрицы A к матрице A, в которой один из элементов aij(x) заменяется на aij(x)+cxd , где

c – символьный коэффициент. Алгоритм пытается с помощью таких продолжений построить продолжение матрицы, которое нарушит сильную цикличность вектора.
Вход алгоритма:
квадратная матрица A с полиномиальными коэффициентами
циклический вектор v
Шаги алгоритма:
Выбрать один из элементов aij матрицы A.
Построить A, получающуюся в результате продолжения aij с символьным коэффициентом: aij=aij+cxd.
Вычислить D – определитель матрицы [Av], являющийся многочленом от переменной x. Если трейлинговый коэффициент D равен c или не зависит от c, то вернуться к шагу 1 и выбрать другой элемент aij. Если больше элементов не осталось, то закончить работу с выдачей результата “v является сильно циклическим”.
Вычислить значение c, отличное от 0 и зануляющее трейлинговый коэффициент D, и подставить его в A.
Если det[Av]=0, то закончить работу с выдачей сообщения “v не является сильно циклическим”, иначе положить A=A и вернуться к шагу 1.

С каждой итерацией алгоритма повышается валюация у определителя матрицы производных в силу системы. В какой-то момент определитель может стать равным нулю, это будет означать, что для данного продолжения вектор не является циклическим. Постепенное зануление этого определителя и является целью алгоритма. К сожалению, в некоторых случаях процесс может продолжаться бесконечно: всегда будет находиться элемент матрицы, продолжение которого увеличит валюацию определителя, при этом этот определитель никогда не станет равным нулю. В этом случае алгоритм не сможет дать содержательный ответ.
Таким образом, данный алгоритм достаточно прост в реализации, но обладает рядом недостатков:
Иногда не может ни доказать сильную цикличность вектора, ни опровергнуть ее.
Результативность алгоритма сильно зависит от выбора элемента в шаге 1.
