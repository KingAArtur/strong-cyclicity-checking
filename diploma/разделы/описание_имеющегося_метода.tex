\section{Описание метода SMP}

Мы рассматриваем задачу проверки сильной цикличности вектора $v$ для заданной системы с матрицей $A$.
А.\,А.~Панфёров в работе~\cite{litPanferov} предложил метод SMP (serial multi-entry prolongation).
Основная идея метода "--- построение продолжения коэффициентов системы,
опровергающего свойство сильной цикличности вектора.
В случае, когда удается построить опровергающее продолжение, мы показываем, что вектор не является сильно циклическим.
В случае же, когда удается доказать, что такое продолжение построить невозможно,
мы можем утверждать, что вектор является сильно циклическим.

Построение такого продолжения производится путем постепенного продолжения отдельных элементов матрицы.
\medskip

Входные данные для метода:
\begin{itemize}
    \item
        квадратная матрица $A$ с полиномиальными коэффициентами;
    \item
        циклический вектор $v$.
\end{itemize}

Метод состоит из следующих шагов:
\begin{enumerate}
    \item
        Некоторым образом выбрать один из~элементов $a_{ij}$ матрицы~$A$.
    \item
        Построить $B$, получающуюся из~$A$ путем продолжения элемента~$a_{ij}$ с~символьным коэффициентом~$c$:
        $b_{ij} = a_{ij} + cx^t$
    \item
        Вычислить $D$ "--- определитель матрицы \diffMatrix[B], являющийся многочленом от~переменной~$x$.
        Если трейлинговый коэффициент~$D$ равен~$c$ или не зависит от~$c$, то вернуться к шагу~1 и выбрать другой элемент~$a_{ij}$.
        Если больше элементов не~осталось, то~закончить работу с~выдачей результата <<$v$ является сильно циклическим>>.
    \item
        Вычислить значение~$c$, отличное от~0 и зануляющее трейлинговый коэффициент~$D$, и подставить его в~$B$.
    \item
        Если $\dett\diffMatrix[B] = 0$, то закончить работу с~выдачей сообщения <<$v$ не~является сильно циклическим>>,
        иначе положить $A := B$ и вернуться к~шагу~1.
\end{enumerate}

С каждой итерацией алгоритма повышается валюация у~определителя матрицы производных в~силу системы.
В какой-то момент определитель может стать равным нулю, это будет означать,
что для построенного продолжения вектор не~является циклическим.
Постепенное зануление этого определителя и является целью алгоритма.

К~сожалению, в~некоторых случаях процесс может продолжаться бесконечно:
всегда будет находиться элемент матрицы, продолжение которого увеличит валюацию определителя,
при~этом этот определитель никогда не~станет равным нулю.
В этом случае алгоритм не~сможет дать содержательный ответ.

Таким образом, данный алгоритм обладает рядом недостатков:
\begin{itemize}
    \item
        Иногда не~может ни~доказать сильную цикличность вектора, ни опровергнуть ее.
    \item
        Результативность алгоритма сильно зависит от~выбора элемента на~шаге~1.
\end{itemize}
