\section{Введение}

Многие задачи из~самых разных областей науки и техники сводятся к~решению систем дифференциальных уравнений.
По~этой~причине решение систем дифференциальных уравнений является достаточно важной проблемой естественных наук.

В~частности, решением дифференциальных уравнений занимается компьютерная алгебра "--- научная дисциплина,
целью которой является поиск точных аналитических решений задач.
Это отличает ее от~традиционной вычислительной математики, решающей задачу численно и находящей лишь приближенный ответ.
Точное аналитическое решение содержит гораздо больше полезной информации о~сути задачи.
\medskip

Одним~из~важных понятий при рассмотрении систем дифференциальных уравнений является понятие циклического вектора.
Циклические векторы для~заданной дифференциальной системы обладают важными свойствами (см.~\cite{litKovacic}),
позволяющими свести задачу решения системы дифференциальных уравнений
к~задаче решения скалярного дифференциального уравнения (например, см.~\cite{litAbramovResolvingSequences}).

Часто в~задачах, связанных с~компьютерной алгеброй, приходится иметь дело со~степенными рядами:
например, они могут выступать в~роли коэффициентов систем дифференциальных уравнений~\cite{litAbramovPowerSeries}.
Сразу~же встает вопрос о~представлении этих потенциально бесконечных структур в~памяти компьютера.
Одним из~традиционных подходов к~решению данной проблемы является работа с~усечениями "--- конечными отрезками рядов, представляющими собой многочлены.
При этом часть информации об~исходной задаче неизбежно теряется.
Так~как мы работаем лишь с~усечениями степенных рядов, а не~со~всем~рядом сразу, перед нами встает вопрос:
останется~ли указанный вектор циклическим при~любом возможном продолжении коэффициентов системы,
или существует такое возможное продолжение, которое нарушает это~свойство?
\medskip

Работа имеет следующую структуру.
В~разделе~2 вводятся основные понятия, относящиеся к~теме усеченных дифференциальных систем и циклических векторов.
В~разделе~3 приводится постановка задачи.
В~разделе~4 формулируются и доказываются некоторые утверждения о~циклических и сильно циклических векторах.
В~разделе~5 рассматривается имеющийся метод проверки сильной цикличности вектора, указываются его~достоинства и недостатки.
В~разделе~6 приводится модифицированный алгоритм,
а в~разделе~7 "--- его программная реализация в~среде компьютерной алгебры Мэйпл.
